\documentclass{bmstu}
\usepackage{xparse}
\usepackage{comment}

\setcounter{tocdepth}{3}
\setcounter{secnumdepth}{3}

\lstset { %    
	language=C++,    
	backgroundcolor=\color{black!5}, % set backgroundcolor    
	basicstyle=\footnotesize,% basic font setting
	}

\begin{document}

\tableofcontents




\chapter{Что такое компьютер}
\section{Немного истории}
Перед тем, как изучать языки программирования стоит понять ....

Для начала ответим на вопрос: "Что такое компьютер" ? Для повседневного обывателя это покажется очевидным, но давайте порасуждаем. 
Если обратится к свободным источникам(например к википедии), то мы увидем множество меканизмов. 
Саммым ранним компьютером я бы назвал антикитерский механизм. 
Всего лишь несколько десятков шестерёнок и магия простых чисел позволяло этому устройству предстакзыать лунные и солнечные затмения, движения небесных объектов и это при условии, что Земля была установлена, как центр отсчёта. 
При таком расположенни планеты, движения всех оставишся объектов перестовало быть простым объединением множества элемтических орбит. 
Так чтоже такое компьютер? Компьютер - это калькулятор. Даже сейчас, спустя полтысячелетия, компьютер остаётся сложным калькулятором, в основе которого стоит сложение, побитывые сдвиги и другие простые операции. 

Рассмотрим простой счётный механизм, состоящий из шетерёнок, рычагов и циферблатов. 
Для начала установим следующие правила: система счисления у нашего компьютера будет позиционной и, для удобства, оставим её десятичной, будет три поля для данных - два для входа и один для выхода. 
Тогда, выствавив два входных числа и использовав набор шестерёнок для передачи момента от входа к выходу, можно легко получить две простейших операции. 



\section{Программа и исполняемый код}




\begin{comment}

\begin{lstlisting}
for (int i = 0; i < iterations; i++)
{
	do something
}
\end{lstlisting}

\end{comment}

\chapter{Основы языка Си}
%история создания
\section{Статические типы данных}
\subsection{Простые типы данных}
В языке Си не так много базовых типов данных, как может показаться с начала. Так как мы помним, что Си --- это более приятный язык для чтения, чем язык ассемблера, то и типов данных в нём не больше. Основные типы --- это:
\begin{itemize}
\item char   -- размером в один байт, хранящий символ из таблицы (ASCII)
\item int    -- размеров в 4 байта, хранящий знаковое число
\item float  -- размеров в 4 байта, хранящий дробное знаковое число
\item double -- тоже самое, что и float, только в размер в два раза больше.
\end{itemize}
Также к эти типы можно модифицировать с помощью слов "long", что даст размер в два раза больше для типа данных int, и unsigned --- которое пометит переменную, как безнаковую.

%разница в инициализации и присвоения.

\subsection{Статические массивы}


\subsection{Текстовые строки}
\subsection{Структуры}
\section{Условные операторы}
\section{Циклы}
\section{Процедуры и функции}
\section{Рекурсия}
\section{Динамические типы данных}
\subsection{Динамические массивы}
\subsection{Списки}
\subsubsection{Очередь}
\subsubsection{Стек}
\section{Задачи}

\chapter{С++}
\section{Отличие Си и С++}
\section{Три столпа ООП}
\section{}



\end{document}

