\documentclass{bmstu}
\usepackage{xparse}
\setcounter{tocdepth}{3}
\setcounter{secnumdepth}{3}
\begin{document}

\tableofcontents




\chapter{Что такое компьютер}
\section{Немного истории}


\chapter{Основы языка Си}
\section{Статические типы данных}
\subsection{Простые типы данных}
\subsection{Статические массивы}
\subsection{Текстовые строки}
\subsection{Структуры}
\section{Условные операторы}
\section{Циклы}
\section{Процедуры и функции}
\section{Рекурсия}
\section{Динамические типы данных}
\subsection{Динамичесике массивы}
\subsection{Списки}
\subsubsection{Очередь}
\subsubsection{Стек}
\section{Задачи}

\chapter{С++}
\section{Отличие Си и С++}
\section{Три столпа ООП}
\section{}



\end{document}

